\documentclass[11pt, aspectratio=169]{beamer}
% \documentclass[11pt,handout]{beamer}
\usepackage[T1]{fontenc}
\usepackage[utf8]{inputenc}
\usepackage{textcomp}
\usepackage{float, afterpage, rotating, graphicx}
\usepackage{epstopdf}
\usepackage{longtable, booktabs, tabularx}
\usepackage{fancyvrb, moreverb, relsize}
\usepackage{eurosym, calc}
\usepackage{amsmath, amssymb, amsfonts, amsthm, bm}
\usepackage[
    natbib=true,
    bibencoding=inputenc,
    bibstyle=authoryear-ibid,
    citestyle=authoryear-comp,
    maxcitenames=3,
    maxbibnames=10,
    useprefix=false,
    sortcites=true,
    backend=biber
]{biblatex}
\AtBeginDocument{\toggletrue{blx@useprefix}}
\AtBeginBibliography{\togglefalse{blx@useprefix}}
\setlength{\bibitemsep}{1.5ex}
\addbibresource{../refs.bib}

\hypersetup{colorlinks=true, linkcolor=black, anchorcolor=black, citecolor=black,
filecolor=black, menucolor=black, runcolor=black, urlcolor=black}

\setbeamertemplate{footline}[frame number]
\setbeamertemplate{navigation symbols}{}
\setbeamertemplate{frametitle}{\centering\vspace{1ex}\insertframetitle\par}

\newcommand{\indep}{\perp\!\!\!\!\perp}


\begin{document}

\title{Inference in Partially Identified Marginal Treatment Effect Models}

\author[Julian Budde]
{
{\bf Julian Budde}\\
{\small University of Bonn}\\[1ex]
}


\begin{frame}
    \titlepage
    \note{~}
\end{frame}

\begin{frame}
    \frametitle{Motivation}

    \begin{itemize}
        \item \citet{mogstad2018using} propose partial identification approach for \textit{extrapolating} information from instrumental variable estimates
        \item Focus is on identification and estimation
        \item Less on \textit{inference}:
        \begin{itemize}
            \footnotesize
            \item Working paper version has an inference procedure \dots
            \item \dots but no simulation results and no implementation.
            \item Authors' \textit{R package} supports bootstrap and subsampling \dots
            \item \dots but ``There does not currently exist a solution for the MTE framework that is both theoretically
            satisfactory and computationally tractable\dots [implemented methods] are known to not be valid in general''
        \end{itemize}
        \item
    \end{itemize}

\end{frame}

\begin{frame}
    \frametitle{This Project}

    \begin{itemize}
        \item Show the standard bootstrap is invalid in simple setting.
        \begin{itemize}
            \item Suggest some alternative approaches for these settings.
        \end{itemize}
        \item Simulation results for coverage probabilities in empirically relevant setting.
        \item Look at inference in linear programs more generally.
    \end{itemize}

\end{frame}

\begin{frame}
    \tableofcontents
\end{frame}

\section{Background}

\begin{frame}
    \frametitle{Marginal Treatment Effect Model: Notation}

    Program evaluation setting:
    \begin{itemize}
        \item Outcome $Y$
        \item Binary Treatment $D$
        \item Potential Outcomes $Y = D Y_{i1} + (1-D) Y_{i0}$
        \item Binary Instrument $Z$
    \end{itemize}

    \vspace{0.5cm}

    \textbf{Treatment Choice Equation}:
    \begin{equation}
        D = I\{p(Z) - U \geq 0\}
    \end{equation}
    where $p(z)$ is the propensity score and $U\sim Unif[0,1]$.

    \vspace{0.5cm}

    $U$ is ``resistance'' to treatment: Small $u$ $\rightarrow$ always take treatment.

\end{frame}

\begin{frame}
    \frametitle{MTE Model: Assumptions}
    $(Y,D,Z)$ are observables, $(Y_1, Y_0, U)$ unobservables.


    \begin{itemize}
        \item $U \indep Z$
        \item $E[Y_d|Z,U] = E[Y_d|U]$ and $E[Y_d^2] < \infty$ for $d \in \{0,1\}$
        \item $U$ is uniform on $[0,1]$ conditional on $Z$.
    \end{itemize}

\end{frame}

\begin{frame}
    \frametitle{MTR Functions}

    Key to the paper: Define everything in terms of (unobservable) MTR functions.

    For $d\in\{0,1\}$:
    \begin{equation}
        m_d(u) \equiv E[Y_d | U=u].
    \end{equation}

    \vspace{0.5cm}

    \textbf{Extrapolation}: Combine
    \begin{itemize}
        \item \textit{Information} on point-identified estimands, with
        \item \textit{Assumptions} on MTR functions.
    \end{itemize}

    \vspace{0.5cm}

    $\Rightarrow$ Linear program to get identified set.

\end{frame}

\section{Simple Example}

\begin{frame}
    \frametitle{Setup}

    \begin{itemize}
        \item Outcome $Y \in [0,1]$.
        \item Binary Treatment $D$
        \item Binary Instrument $Z$
        \item Propensity score: $p(0) = 0.4$ and $p(1) = 0.6$.
    \end{itemize}

    \vspace{0.5cm}

    \textbf{Identify LATE}: $\beta_0^s = E[Y_1 - Y_0 | u \in (0.4, 0.6]]$.

    \vspace{0.5cm}

    \textbf{Target Parameter}: $\beta^* = E[Y_1 - Y_0 | u \in (0.4, 0.8]]$.

\end{frame}

\begin{frame}
    \frametitle{Solution}
    The linear programs have an explicit solution:
    \begin{equation}
        \beta^* \in [\omega \beta_0^s + (1 - \omega) * (-1), \omega \beta_0^s + (1-\omega) * 1],
    \end{equation}
    where $\omega = \frac{p(1) - p(0)}{\overline{u} + p(1) - p(0)}$ is the relative complier share.
\end{frame}

\begin{frame}
    \frametitle{Solution with Constraints}

    We can assume that MTR functions must be \textit{increasing} in $u$.
    This introduces a \textit{kink} in the solution.

    \begin{equation}
        \overline{\beta^*}(\beta_s)=
        \begin{cases}
            \omega \beta_s + (1 - \omega),& \quad \text{if } \beta_s \geq 0\\
            \beta_s + (1 - \omega),              & \quad \text{if } \beta_s < 0,
        \end{cases}
    \end{equation}
    and
    \begin{equation}
        \underline{\beta^*}(\beta_s)=
        \begin{cases}
            \beta_s - (1 - \omega),& \quad \text{if } \beta_s \geq 0\\
            \omega \beta_s - (1 - \omega),              & \quad \text{if } \beta_s < 0.
        \end{cases}
    \end{equation}

\end{frame}

\begin{frame}
    \frametitle{Inference}

    From~\cite{fang2019inference} we know the standard \textit{bootstrap fails} when $\phi(\theta)$ is not fully differentiable.

    \vspace{0.5cm}

    \textbf{Alternatives}: Based on \textit{directional} differentiability of $\phi$ we can use adjusted delta methods.
    \begin{itemize}
        \item \textit{Analytical} delta method, e.g.~\cite{fang2019inference}.
        \item \textit{Numerical} delta method, e.g.~\cite{hong2018numerical}.
    \end{itemize}


\end{frame}

\begin{frame}
    \frametitle{Analytical delta bootstrap}:
    \begin{itemize}
        \item[1.] Bootstrap approximation to distribution of $\sqrt{n}(\hat{\beta^s} - \beta_0^s)$.
        \item[2.] Suitable estimator $\hat{\phi'_n}$ for directional derivative.
        \item[3.] Delta method:
        \begin{equation*}
            \hat{\phi'_n}(\sqrt{n}(\hat{\beta^s}^*_n - \hat{\beta^s}_n))
        \end{equation*}
    \end{itemize}

    In our case: Simple pre-test
    \begin{equation*}
        \hat{\phi'_n}(h)=
        \begin{cases}
            h  \omega,& \quad \text{if } \sqrt{n}\hat{\beta^s}/\hat{\sigma^s} > \kappa_n\\
            I(h < 0)  h + I(h > 0)  h  \omega,& \quad \text{if } |\sqrt{n}\hat{\beta^s}/\hat{\sigma^s}| \leq \kappa_n\\
            h,& \quad \text{if } \sqrt{n}\hat{\beta^s}/\hat{\sigma^s} < -\kappa_n\\
        \end{cases}
    \end{equation*}
    where we require $\kappa_n \to \infty$ but more slowly than $\sqrt{n}$, i.e. $\kappa_n / \sqrt{n} \to 0$.
\end{frame}

\begin{frame}
    \frametitle{Numerical delta bootstrap}

    If we don't know the analytical structure of $\phi'$, we can use a numerical approximation:
    \begin{equation*}
        \hat{\phi'}_{n, s_n} \equiv \frac{1}{s_n}\{\phi(\hat{\beta^s} + s_n h) - \phi(\hat{\beta^s})\}
    \end{equation*}

    Combining this with a bootstrap approximation to $\sqrt{n}(\hat{\beta^s} - \beta_0^s)$ we get
    \begin{equation*}
        \hat{\phi'}_{n, s_n}(\sqrt{n}(\hat{\beta^s} - \beta_0^s)) \equiv \frac{1}{s_n}\{\phi(\hat{\beta^s} + s_n \sqrt{n}(\hat{\beta^s} - \beta_0^s)) - \phi(\hat{\beta^s})\},
    \end{equation*}
    the distribution of which we can use to construct confidence intervals.
    We require $s_n\to0$ but $s_n\sqrt{n} \to \infty$.
\end{frame}

\begin{frame}
    \frametitle{Monte Carlo Simulations}

    \begin{itemize}
        \item True parameter $\beta^*$ at upper bound of identified set.
        \item Compare: Standard bootstrap, analytical and numerical delta method.
    \end{itemize}

    \vspace{0.5cm}

    To construct $1-\alpha = 0.95$ confidence interval for \textit{true parameter}:
    \begin{itemize}
        \item Use asymptotic approximations from above.
        \item Construct one-sided $\alpha/2$ intervals for the upper and lower bound.
        \item By~\cite{imbens2004confidence} logic these should be \textit{conservative}.
    \end{itemize}

    \vspace{0.5cm}

    Setting: $N_{boot} = N_{sims} = 250$, $s_n = 1 / \sqrt{n}$, $\kappa_n = \sqrt{n}$.

\end{frame}

\begin{frame}
    \frametitle{Results: Coverage}

    \begin{figure}
		% \includegraphics[width=\textheight]{../bld/boot/figures/coverage.png}
        \caption{Coverage by Method and True Parameter}
    \end{figure}

\end{frame}

\begin{frame}
    \frametitle{Outlook}
    \begin{itemize}
        \item[1.] Get inference to work: CI construction, tuning params, check analytical delta.
        \item[2.] Understand solutions under different constraints: Shape restrictions, parametric restrictions (written package for this)
        \item[3.] Simulate bootstrap and numerical delta method in
        \begin{itemize}
            \item[(a)] More complicated example $\Rightarrow$ original paper example
            \item[(b)] Empirically relevant example $\Rightarrow$ literature
        \end{itemize}
        \item[4.] Justify numerical delta bootstrap theoretically by studying characteristics of LP solutions
    \end{itemize}

\end{frame}

\begin{frame}[allowframebreaks]
    \frametitle{References}
    \renewcommand{\bibfont}{\normalfont\footnotesize}
    \printbibliography
\end{frame}

\end{document}
