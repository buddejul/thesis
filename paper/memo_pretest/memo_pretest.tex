% Thesis guidelines:
% - 40 pages including graphs, graphics and pictures, TOC/appendix/bibliography not counted
% - 1.5 line-spacing
\documentclass[12pt,a4paper,english]{article} %document type and language

\usepackage[onehalfspacing]{setspace}
% \linespread{1.5}

\usepackage[utf8]{inputenc}	% set character set to support some UTF-8
\usepackage{babel} 	% multi-language support
% \usepackage{sectsty}	% allow redefinition of section command formatting
\usepackage{tabularx}	% more table options
\usepackage{titling}	% allow redefinition of title formatting
\usepackage{imakeidx}	% create and index of words
\usepackage{xcolor}	% more color options
\usepackage{enumitem}	% more list formatting options
\usepackage{tocloft}	% redefine table of contents, new list like objects

\usepackage[centering,noheadfoot,left=3cm, right=2cm, top=2cm, bottom=2cm]{geometry}

%set TOC margins
\setlength{\cftbeforesecskip}{15pt} % skip in TOC

% remove paragraph white space and modify space between list items
\usepackage{parskip}

% Set font globally
\usepackage{lmodern}                % load Latin modern fonts
\usepackage[defaultsans]{cantarell} % cantarell fonts

% HACK: https://tex.stackexchange.com/questions/58087/how-to-remove-the-warnings-font-shape-ot1-cmss-m-n-in-size-4-not-available
\usepackage{anyfontsize}

% set LaTeX global font
\renewcommand{\familydefault}{\sfdefault}
\renewcommand{\sfdefault}{lmss}

% set styling headings
%\allsectionsfont{\usefont{OT1}{phi}{b}{n}}

\usepackage[capposition=top]{floatrow}

\usepackage{float} 	% floats
\usepackage{graphicx}	% Graphics
\usepackage{amsmath}	% extensive math options
\usepackage{amssymb}	% special math symbols
\usepackage[Gray,squaren,thinqspace,thinspace]{SIunits} % elegant units
\usepackage{listings}                                   % source code

% Custom Operators
%% Expectation symbol
\DeclareMathOperator*{\E}{\mathbb{E}}
\DeclareMathOperator*{\Cov}{\mathrm{Cov}}
\DeclareMathOperator*{\Var}{\mathrm{Var}}

% missing math commands
\providecommand{\abs}[1]{\left\lvert#1\right\rvert}                    % |.|
\providecommand{\br}[1]{\left(#1\right)}                               % (.)
\providecommand{\sbr}[1]{\left[#1\right]}                              % [.]
\providecommand{\ddfrac}[2]{\frac{\displaystyle #1}{\displaystyle #2}}
% use \math rm{d} to include math differential

% independence symbol
% https://tex.stackexchange.com/questions/79434/double-perpendicular-symbol-for-independence
\newcommand{\indep}{\perp\!\!\!\!\perp}


% options for listings
\lstset{
  breaklines=true,
  postbreak=\raisebox{0ex}[0ex][0ex]{\ensuremath{\color{red}\hookrightarrow\space}},
  numbers=left,
  numbersep=5pt,
  numberstyle=\tiny\color{gray},
  basicstyle=\footnotesize\ttfamily
}

% NEEDS to be before hyperref, cleveref and autonum
% number figures, tables and equations within the sections
\numberwithin{equation}{section}
% \numberwithin{figure}{section}
% \numberwithin{table}{section}

% references and annotation, citations
\usepackage[small,bf,hang]{caption}        % captions
\usepackage{subcaption}                    % adds sub figure & sub caption
\usepackage{sidecap}                       % adds side captions
\usepackage{hyperref}                      % add hyperlinks to references
\usepackage[noabbrev,nameinlink]{cleveref} % better references than default~\ref
% Hack:https://tex.stackexchange.com/questions/285950/package-autonum-needs-the-obsolete-etex-package
\expandafter\def\csname ver@etex.sty\endcsname{3000/12/31}
\let\globcount\newcount
% Deactivate for now to avoid issues with equation* environment.
% \usepackage{autonum}                       % only number referenced equations
\usepackage{url}                           % URLs
% Biblatex throws error when cite is used.
% Similar warning is given by natbib.
% \usepackage{cite}                          % well formed numeric citations

% % biblatex for references
% \usepackage{biblatex}
% \addbibresource{literature.bib}
% % csquotes recommended: https://tex.stackexchange.com/questions/229638/package-biblatex-warning-babel-polyglossia-detected-but-csquotes-missing
% \usepackage{csquotes}
% \addbibresource{refs.bib}

% https://tex.stackexchange.com/questions/144764/author-year-citation-in-latex
\usepackage[round]{natbib}

% Avoid space before footnotes when \footnote{...} is on next line.
% https://tex.stackexchange.com/questions/94563/new-line-for-footnote-without-blank-space
\usepackage{xpatch}
\xpretocmd{\footnote}{\unskip}{}{}

% format hyperlinks
\colorlet{linkcolour}{black}
\colorlet{urlcolour}{blue}
\hypersetup{colorlinks=true,
            linkcolor=linkcolour,
            citecolor=linkcolour,
            urlcolor=urlcolour}

%\usepackage{todonotes} % add to do notes
\usepackage{epstopdf}  % process eps-images
\usepackage{float}     % floats
\usepackage{fancyhdr}  % header and footer
% HACK: https://tex.stackexchange.com/questions/664532/fancyhr-warning-footskip-is-too-small
\setlength{\footskip}{15pt}

% default path for figures
\graphicspath{{figures/}}

% If we have multiple directories, specify them like this: \graphicspath{{figures_ch1/}{figures_ch2/}}.

% For rendering tikz
\usepackage{pgfplots}
\pgfplotsset{compat=1.18}
\usetikzlibrary{decorations.pathreplacing} % Load the library for drawing braces


% Define some math environments
\usepackage{amsthm}

\newtheorem{theorem}{Theorem}[section]
\newtheorem{corollary}{Corollary}[theorem]
\newtheorem{lemma}[theorem]{Lemma}

\theoremstyle{definition}
\newtheorem{definition}{Definition}[section]

\theoremstyle{remark}
\newtheorem*{remark}{Remark}

\theoremstyle{plain}
\newtheorem{assumption}{Assumption}

% https://tex.stackexchange.com/questions/24840/use-courier-font-inline-on-text
\usepackage{courier}

% https://tex.stackexchange.com/questions/639234/how-to-put-braces-over-certain-parts-of-matrices
\usepackage{nicematrix}


% set header and footer
\pagestyle{fancy}
\fancyhf{}                           % clear all header and footer fields
\cfoot{\thepage}                     % add page number
\renewcommand{\headrulewidth}{0pt} % add horizontal line of 0.4pt thick

\title{Memo: Pretesting for Non-differentiable functionals}
\author{Julian Budde}
\date{\today}
\begin{document}

\maketitle

\section{Setting}
We want to perform inference on the function $\phi(\theta) = \min\{0, \theta\}$ where $\theta \in \mathbb{R}$.
We have an estimator $\hat{\theta} = Y$ where $Y \sim N(\theta, \frac{\sigma^2}{n})$.
Finite sample normality is used as a replacement for asymptotic approximations.
Our goal is to construct a two-sided confidence interval with coverage probability at least $1-\alpha$.

Studying $\phi(\theta) = \min\{0, \theta\}$ can be motivated, for example, by considering inference on the optimal value of a linear program.
Consider the problem $\min c_1 x_1 + c_2 x_2$ over $x\in[0,1]^2$ where $c_2\neq0$ is treated as known.
Then the optimal value ($c'x$ evaluated at the (or an) optimal solution), is given by $\phi(c_1)$.
The inference problem arises from treating $c_1$ as unknown.

\paragraph{Finite Sample Distribution}
We can characterize the exact finite sample distribution of $\sqrt{n}(\phi(Y) - \phi(\theta))$ as follows:
\begin{equation}
  G^f(\theta)=
    \begin{cases}
      \sqrt{n}(\min\{Y, 0\} - 0) = \min\{N(\sqrt{n}\theta, \sigma^2), 0\}& \text{if } \theta > 0,\\
      \sqrt{n}(\min\{Y, 0\} - 0) = \min\{N(0,1), 0\}& \text{if } \theta = 0,\\
      \sqrt{n}(\min\{Y, 0\} - \theta) =\min\{N(0,1), -\sqrt{n}\theta\}& \text{if } \theta < 0,\\
  \end{cases}
\end{equation}
Importantly, this distribution is discontinuous in the unknown parameter $\theta$.

\paragraph{Asymptotic Approximation}
Typically, we build an estimator based on the following asymptotic approximation:
\begin{equation}
  G^a(\theta)=
    \begin{cases}
      0 & \text{if } \theta > 0,\\
      \min\{N(0,1), 0\}& \text{if } \theta = 0,\\
      N(0,1)& \text{if } \theta < 0.\\
  \end{cases}
\end{equation}

The asymptotic distribution coincides with the finite sample one when $\theta=0$.
When $\theta\neq0$, it is easy to see, that $G$ converges to $G^a$ when $n\to\infty$.
However, when $\theta$ is small relative to the sample size, there is a large difference between asymptotic approximation and finite sample distribution.
For example, consider $\theta < 0$: The asymptotic approximation is $N(0,1)$, while the finite sample distribution is $N(0,1)$ only to the left of $\sqrt{n}\theta$ and has the remaining mass at $-\sqrt{n}\theta$.
If $\sqrt{n}\theta$ is close to $0$, these distributions are very different.
For example, the $1-\alpha/2$ quantiles are $z^a_{1-\alpha/2} = 1.96$ and $z^f_{1-\alpha/2} = -\sqrt{n}\theta$ if $\alpha=0.05$.

\paragraph{Estimator}
A pretest estimator can be constructed as follows (cf., for example,~\cite{fang2019infdirdiff}):
\begin{equation}
  \hat{G}^a(\theta)=
    \begin{cases}
      0 & \text{if } \frac{\sqrt{n}\hat{\theta}_n}{\sigma^2} > \kappa_n,\\
      \min\{N(0,1), 0\}& \text{if } |\frac{\sqrt{n}\hat{\theta}_n}{\sigma^2}| = \kappa_n,\\
      N(0,1)& \text{if } \frac{\sqrt{n}\hat{\theta}_n}{\sigma^2} < - \kappa_n.\\
  \end{cases}
\end{equation}
Here, $\kappa_n$ is a sequence of positive numbers with $\kappa_n \to \infty$ but $\frac{\kappa_n}{\sqrt{n}} \to 0$.
Then, under $H_0: \theta = 0$, this test is cosnsistent in the sense that we select the right asymptotic distribution with probability approaching 1 as $n\to\infty$.

~\cite{fang2019infdirdiff}, for example, suggest to choose $\kappa_n = \Phi^{-1}(1-\frac{\tilde{\alpha_n}}{2})$ where $r_n\to 0$ slow enough.
This means, that the probability of a Type I error converges to zero at an appropriate rate.
In what follows, we will hence pick $\kappa_n = \Phi^{-1}(1-\frac{\tilde{\alpha}}{2})$ for some fixed value $\tilde{\alpha}$.

The goal in the following sections is to derive the exact finite sample coverage probabilities of a confidence interval based on this pretest as a function of $\tilde{\alpha}$.

\section{Finite Sample Coverage Probabilities}

We first consider the case $\theta = 0$ and then turn to a discussion of $\theta < 0$ and $\theta > 0$ (where we think of them being close to zero).
We denote the coverage probability of a two-sided confidence interval based on the procedure above by $p(\tilde{\alpha}, \theta)$, to make explicit the dependence on the unknown $\theta$ and the tuning parameter $\tilde{\alpha}$.

\paragraph{Case $\theta = 0$}
To derive the exact coverage probabilities we make use of the normal approximation $Y\sim N(\theta, \frac{\sigma^2}{n})$.
We will denote probabilities under $\theta = t$ by $P_t(\cdot)$, so $P_0(\cdot)$ corresponds to this first case.

The covergae probabilities are derived by conditioning on the result of the pretest.
We then calculate exact conditional coverage probabilities by noting that conditioning on pretests, $\sqrt{n}Y/\sigma$ behaves like a truncated normal variable.

The result is given by

\begin{align*}
  p(0, \tilde{\alpha}) & =  \tilde{\alpha}/2 \times 1 \\
  & + (1 - \tilde{\alpha}) \times \left(\frac{1}{2} \times 1 + \frac{1}{2} \min\left\{1, 1 - \frac{\alpha/2 - \tilde{\alpha}/2}{1/2 - \hat{\alpha}/2}\right\}\right) \\
  & + \tilde{\alpha}/2 \times \max\left\{0, 1 - \frac{\Phi(\frac{\alpha}{2})}{\Phi(\frac{\tilde{\alpha}}{2})}\right\}.
\end{align*}
The first line corresponds to rejecting in favor of $\theta > 0$, the second to not rejecting $\theta = 0$ and the last line to rejecting in favor of $\theta < 0$.

Some intuition:
\begin{itemize}
  \item Conditional on $\sqrt{n}Y/\sigma > \Phi^{-1}(1-\tilde{\alpha}/2)$, we have $\phi(Y) = 0$ and we use the asymptotic approximation $\hat{G}^a(\theta) = 0$. Hence the CI collapses to the point $0$ and coverage is $100\%$.
  \item Conditional on $\sqrt{n}Y/\sigma < -\Phi^{-1}(1-\tilde{\alpha}/2)$, we use the $N(0,1)$ approximation for quantiles. However, we introduce a distortion by pre-testing: In the extreme case when $\tilde{\alpha} \leq \alpha$ we condition on $Y$ values so small, such that we never cover $\phi(\theta)=0$.
  \item The pre-testing distortion is more delicate when we don't reject $\theta=0$: With probability 0.5 under $\theta=0$, we will have $Y>0$. In this case we cover with probability one since the CI is the point $0$.
  When $Y<0$, we have a distortion by conditioning on small $Y$ values. When the pre-test is not too conservative, so $\tilde{\alpha} > \alpha$, we cover with probability one, since we condition on being in a subset of the CI.
  When the pre-test is conservative, so $\tilde{\alpha} \leq \alpha$, we cover with probability in $[1-\alpha, 1]$.
\footnote{The exact result follows from the CDF of a truncated normal: In the extreme, when $\tilde{\alpha} = 0$, the conditional distribution is $N(0,1)$ truncated at $0$.
  Hence, the probability to fall below the $z_{\alpha/2}$ quantile is $2\times\alpha/2 = \alpha$.
}
\end{itemize}

\bibliographystyle{../abbrvnat_my_version}
\bibliography{../refs.bib}

\end{document}
